
The key idea of using the Fourier Transform for image compression --- to move pixel data from the intensity domain (0 to 255) into the frequency domain and then limit the number of terms to reduce space --- has been implemented in both our DFT and our Cooley-Tukey algorithms.

Moreover, our original calculations stated that Cooley-Tukey had much lower complexity than the naive implementation, and we have demonstrated this conclusion by writing and testing both algorithms, showing that that the Cooley-Tukey FFT is both far faster than and equally accurate to the naive DFT.

Since our implementations are unoptimized, they are still limited in that they are slow, insufficiently compress images, still produce many artifacts after encoding and decoding. We leave adding heuristics to get comparable performance to industry standards for future work.
