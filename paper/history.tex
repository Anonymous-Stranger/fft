The \textit{Fast Fourier Transform} could not exist without the original Fourier Transform it originated from. In 1807 French mathematician Jean Baptiste Joseph Fourier was the first to define the original Fourier transform and its inverse:
\begin{align*}
F_c(u) &= \dfrac{2}{\pi}\int_{0}^{\infty} f(x) cos(ux)dx\\
f(x) &= \int_{0}^{\infty}F_c(u)cos(ux)du
\end{align*}

These equations are commonly named the cosine Fourier transform and the inverse cosine Fourier transform, respectively. From these equations the general Fourier Transform was defined using the imaginary exponentiation used Euler's formula.
\begin{align*}
F(u) &= \int_{-\infty}^{\infty}e^{-ixu}dx\\
f(x) &= \dfrac{1}{2\pi}\int_{-\infty}^{\infty}F(u)e^{-ixu}du
\end{align*}
These equations can be used to accurately shift continuous formulae to and from the frequency domain.

The major issue with this final equation is that it is limited to continuous functions. By converting from an integral sum to a summation of the terms, and adding a new factor, $W_N$, mathematicians were able to create a discrete version of the equation.
\begin{align*}
C(k) &= \sum_{n = 0}^{N-1}X(n)W_N^{nk}\\
X(n) &= f(n/N)\\
W_N &= e^{-2\pi/N}
\end{align*}
This equation will be discussed in depth later on in the report, but it is important to note that a sum such as this requires O($N^2$) computational time to compute. This becomes a major limitation when trying to utilize this process on large sets of data such as photos or sound files.

In 1965 a pair of men, J.W. Cooley and J.W. Tukey, derived a new process for this transform that required far less computational time: O($nlogn$). They were not the first to design a formula to solve this equation efficiently, but other earlier permutations only worked on sets of very specific lengths. Cooley and Tukey are therefore credited with the FFT algorithm that will be explored in this report, since they were the first to create this efficient transfrom algorithm for any number of discrete points.

Or were they? In 1977 mathematician H.H. Goldstine dug up  old reports written by the famous mathematician Carl Friedrich Gauss in 1805. In one of these reports Gauss needed to interpolate the orbit of asteroids Pallas and Juno. Not only did Gauss create the first fourier transform to convert the paths of these orbiting rocks to the frequency domain, he actually stumbled upon nearly the same FFT algorithm described in the 1965 Cooley and Tukey paper. The major difference between the two being the description of the "Twiddle Factor", $W_N$. Despite this discovery of Gauss's hidden work, the FFT had already been named and credited to Cooley and Tukey, yet  a few members of the field still prefer to call it the GFT, the Gauss Fourier Transform, to this day.
